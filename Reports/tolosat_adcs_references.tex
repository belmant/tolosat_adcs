\documentclass{article}
\usepackage[utf8]{inputenc}
\usepackage{amsmath,amsfonts,amssymb,stmaryrd, dsfont}
\usepackage[a4paper]{geometry}
\usepackage[english]{babel}
%\usepackage[T1]{fontenc}

\usepackage{graphicx}
%\usepackage{tikz-qtree}

%pseudocode
%\usepackage{algorithm}
%\usepackage{algpseudocode}

% \usepackage{minted}
\usepackage{bm}
\usepackage{xcolor}
\usepackage{graphicx}
\definecolor{LightGray}{gray}{0.95}

\usepackage{array}
\newcolumntype{C}[1]{>{\centering}m{#1}}

\usepackage{hyperref}
\hypersetup{
  colorlinks = true,
  linkcolor  = black,
  citecolor = blue
}

\newcommand{\norm}[1]{\left\lVert#1\right\rVert}
\newcommand{\Iintv}[2]{\llbracket #1 , #2 \rrbracket}

\renewcommand\arraystretch{2} % vertical array spacing
\setlength{\tabcolsep}{20pt} % horizontal array spacing

% \newcommand{\mintedcpp}{\begin{minted}[bgcolor=LightGray,fontsize=\footnotesize,linenos]{cpp}}
% \newcommand{\mintinpy}{\mintinline{python}}
% \newcommand{\mintedpy}{\begin{minted}[bgcolor=LightGray,fontsize=\footnotesize,linenos]{python}}

\title{TOLOSAT - ADCS references}
\author{Cédric Belmant}
\date{\today}

\begin{document}

\maketitle
\abstract{The objective of this document is to give a basis for the coordinate systems used in the frame of the Tolosat project. All used frames shall be listed here and explained, to be potentially reused by any other subsystem.}
\tableofcontents

\newpage
\part{Reference frames}
Here we list all the reference frames and notations used in the frame of the Tolosat project.

\section{Descrption of the Earth-based, satellite-based and body frames}

The $ECI$ (Earth-Centered Inertial) frames are Earth-centered gallilean frames. The one we will use is the J2000 frame, as defined below. \\\\
Various satellite-based frames exsist, such as the $RSW$ frame that uses the radial, along-track and cross-track vectors to describe a movement in a non-gallilean frame from the body's inertial center \cite{vallado2004covariance}. We will use this satellite-based frame for the project. \\\\
The body frame is the reference frame for the satellite, in which he appears motionless. It shares its origin with the $RSW$ frame but the axis are arbitrarily chosen and rotate with the satellite. \\\\THIS SECTION NEEDS TO BE COMPLETED.

\subsection{J2000}

The \href{https://en.wikipedia.org/wiki/Earth-centered_inertial}{J2000} frame is widely used to describe the position of celestial objects in a fixed referential. We will use it for the following: \\
\begin{itemize}
\item Tolosat's orbital dynamics
\item Sun position for power computations \\
\end{itemize}

As described in \cite{book:911134}, the first axis ($X_{J2000}$) points towards the vernal equinox, the second one ($Y_{J2000}$) is rotated 90 degrees towards the East from the first one, and the third one ($Z_{J2000}$) directed towards the North Pole thus completing the frame.

\subsection{RSW}

The RSW frame, as described in \cite{vallado2004covariance} and \cite{book:911134}, is a satellite coordinate system usually used for the description of relative positions or displacements. The first axis ($R$) is the normalized radius vector of the satellite with respect to the Earth's center; the third axis ($W$) is the normalized cross-product between $R$ and the satellite velocity vector (normal w.r.t. the orbital plane), and the second one ($S$) completes the orthonormal RSW frame (along the direction of the velocity vector, and in its exact direction if the orbit is circular or if the spacecraft is on the apogee or perigee of an elliptic orbit). \\
By noting $\vec{r}$ the spacecraft position vector and $\vec{v}$ its velocity in the J2000 frame, we obtain:
\[ R = \frac{\vec{r}}{\norm{\vec{r}}}, \;\; W = \frac{\vec{r}\wedge \vec{v}}{\norm{\vec{v}}}, \; \; S = W \wedge R \]

\subsection{Body frame}

TO BE COMPLETED

\section{Coordinate transformations}

To transform the position, velocity and acceleration vectors from an inertial frame $\bm{I}$ to a translated rotating frame $\bm{R}$, it is necessary to accomodate for some corrections regarding the velocity and acceleration vectors. For the position, we only need to multiply the position by the transition matrix:

\begin{equation}
\vec{r}_I = P_{I \rightarrow R}\vec{r}_R
\end{equation}

with $P_{I \rightarrow R}$ the transition matrix from $\bm{I}$ to $\bm{R}$ assembled by putting the $\bm{R}$ base vectors coordinates (in $\bm{I}$) on its columns. Furthermore, as $P_{I \rightarrow R}$ is orthogonal:

\begin{equation}
\vec{r}_R = P^\top_{I \rightarrow R}\vec{r}_I
\end{equation}

As we can describe the transformation as a translation and a rotation from the inertial frame, for the velocity and acceleration we obtain \cite{book:911134}: \\

\begin{equation}
  \vec{v}_I = \underbrace{\vec{v}_R}_{\text{velocity in }\bm{R}} + \underbrace{\vec{\omega}_{R/I}\wedge \vec{r}_R}_{\text{rotation velocity}} + \underbrace{\vec{v}_{origin}}_{\text{origin velocity}}
\end{equation}
\begin{equation}
  \vec{a}_I = \underbrace{\vec{a}_R}_{\text{acceleration in }\bm{R}} + \underbrace{ \dot{\vec{\omega}}_{R/I}\wedge\vec{r}_R + \vec{\omega}_{R/I}\wedge \left( \vec{\omega}_{R/I}\wedge\vec{r}_R \right) + 2\vec{\omega}_{R/I}\wedge\vec{v}_R}_{\text{rotation acceleration}} + \underbrace{\vec{a}_{origin}}_{\text{origin acceleration}}
\end{equation}

where $\vec{\omega}_{R/I}$ and $\dot{\vec{\omega}}_{R/I}$ are respectively the rate of rotation of $\bm{R}$ w.r.t. $\bm{I}$ (representing its angular velocity) and its derivative (representing its angular acceleration). Using these relations we can easily obtain the transformed velocity and acceleration in the translated rotating frame $\bm{R}$.

\section{Attitude representation}

Attitude representation is done using quaternions. Quaternions are dimension 4 elements (also called hypercomplex numbers), that are written as the following \cite{QUA}:

\begin{equation} Q = q_0\textbf{e} + q_1\textbf{i} + q_2\textbf{j} + q_3\textbf{k}
\end{equation}
with $\textbf{i}$, $\textbf{j}$, $\textbf{k}$ such that :
\[ \textbf{i}\textbf{j} = \textbf{j}\textbf{k} = \textbf{k}\textbf{i} = -\textbf{j}\textbf{i} = -\textbf{k}\textbf{j} = - \textbf{i}\textbf{k} \; ; \; \; \; \textbf{i}^2 = \textbf{j}^2 = \textbf{k}^2 = \textbf{-e}
\]

Euler angles, that we will not explain here, are another commonly used set of reprentations that describe rotations, but these representations yield a singularity that can be problematic for numerical computations, hence our use of quaternions.


\subsection{Main properties of quaternions}

Quaternions algebra allows us to represent the cross product as defined in $\mathbb{R}^3$, characterized by the anti-commutativity described above. $Q$, $q_0\textbf{e}$ is its real part, while $q_1\textbf{i} + q_2\textbf{j} + q_3\textbf{k}$ is its pure part. The pure part of a quaternion is always identifiable to a vector in  $\mathbb{R}^3$. By analogy with the complex numbers, we denote the conjugate a quaternion $Q$ by
\[ \overline{Q} = q_0\textbf{e} - q_1\textbf{i} - q_2\textbf{j} - q_3\textbf{k}
\]

To represent $\mathbb{R}^3$ rotations, only a certain subset of quaternions is used: the set $S$ of norm 1 quaternions (i.e. quaternions $Q \in \mathbb{H}$ such that $q_0^2 + q_1^2 + q_2^2 + q_3^2 = 1$). Those are called \textbf{attitude quaternions}.
These quaternions can then be expressed as follows:

\begin{equation} Q = \cos\left(\frac{\theta}{2}\right) + \sin\left(\frac{\theta}{2}\right)\vec{\Delta}
\end{equation}

There is a group homomorphism between $S$ and the set of $\mathbb{R}^3$ rotations, in other words any rotation in $\mathbb{R}^3$ can be represented by an attitude quaternion and reciprocally. An attitude quaternion that has the same kind of expression as above is associated to a rotation of axis $\vec{\Delta}$ and angle $\theta$. \\
This analogy between quaternions and rotations relies on the Euler theorem, that states \cite{ADC} that any rotation from a given frame to another can be described as a single rotation along a unique axis common to both frames. \\

\subsubsection{Quaternion derivative and angular velocity}

The angular velocity of a movement is intrinsic to this one, and consequently it does not change from one referential to another. The only difference is that its expression in different frames is obtained via different basis. We will denote by $\bm{I}$ and $\bm{R}$ respectively a given inertial frame and a rotating frame, of basis $(XYZ)$ and $(xyz)$.

\noindent The quaternion derivative w.r.t. time is expressed as follows \cite{QUA}:
\begin{equation}
\dot{Q} = \frac{1}{2}\overline{Q}\vec{\omega}_{R/I} \; \; \; \; ; \; \; \; \; \vec{\omega}_{R/I} = p\textbf{i} + q\textbf{j} + r\textbf{k}
\end{equation}
where $\vec{\omega}_{R/I}$ is expressed as a pure quaternion, that yields in the rotating basis (as $\mathbb{R}^3$ vector):
\begin{equation}
\vec{\omega}_{R/I} = p\vec{x} + r\vec{y} + r\vec{z}
\end{equation}

\noindent This allows us to calculate the derivative of a quaternion with respect to time by knowing its angular velocity, and vice-versa. Let us note that the equation above involves quaternions products. However we can calculate $\dot{Q} = \frac{1}{2}\overline{Q}\vec{\omega}_{R/I}$ in matrix form by setting:

\begin{equation}
\dot{Q} = \begin{bmatrix} \dot{q}_0 \\ \dot{q}_1 \\ \dot{q}_2 \\ \dot{q}_3 \end{bmatrix} = \frac{1}{2}\overline{Q}\vec{\omega}_{R/I} = \frac{1}{2}M_SQ = \begin{bmatrix} 0 & -p & -q & -r \\
p & 0 & r & -q \\ q & -r & 0 & p \\ r & q & -p & 0 \end{bmatrix} \begin{bmatrix} q_0 \\ q_1 \\ q_2 \\ q_3 \end{bmatrix}
\end{equation}

\subsubsection{Attitude quaternion and rotation matrix}

Once that we determined the attitude quaternion, the transition from the rotating frame to the inertial frame is done via the following $P_{I \rightarrow R}$ transition matrix \cite{QUA}:

\[
P_{I \rightarrow R} = \begin{bmatrix} 2(q_0^2 + q_1^2)-1 & 2(q_1q_2 - q_0q_3) & 2(q_1q_3 + q_2q_0) \\ 2(q_1q_2 + q_0q_3) & 2(q_0^2 + q_2^2)-1 & 2(q_2q_3 - q_1q_0) \\ 2(q_1q_3 - q_2q_0) & 2(q_2q_3 + q_1q_0) & 2(q_0^2 + q_3^2) -1\end{bmatrix}
\]


%
% \section{Notations}
% Throughout this study, we will use the following notations: \\ \\
% Primary and secondary objects positions and speeds at $t_k$ are stored respectively in $X_p^k$ and $X_s^k$ as follows
% $$ \bm{X_p^k} = \begin{bmatrix} x_p^k \\ y_p^k \\ z_p^k \\ \dot{x_p^k} \\ \dot{y_p^k} \\ \dot{z_p^k} \end{bmatrix}_{ECI} = \begin{bmatrix} \bm{r_p^k} \\ \bm{v_p^k} \end{bmatrix}_{ECI}
%  \text{ and }
%  \bm{X_s^k} = \begin{bmatrix} x_s^k \\ y_s^k \\ z_s^k \\ \dot{x_s^k} \\ \dot{y_s^k} \\ \dot{z_s^k} \end{bmatrix}_{ECI} = \begin{bmatrix} \bm{r_s^k} \\ \bm{v_s^k} \end{bmatrix}_{ECI}
% $$
% The relative positions and speeds at $t_k$ are stored in $Y_k$ defined as
% $$\bm{Y}^k = \bm{X}_s^k - \bm{X}_p^k = \begin{bmatrix} \bm{R}^k \\ \bm{V}^k \end{bmatrix} = \begin{bmatrix} \bm{r_s^k - r_p^k} \\ \bm{v_s^k - v_p^k} \end{bmatrix}_{ECI}
%  \text{ where } \bm{R}^k = \begin{bmatrix} R^k_1 \\ R^k_2 \\ R^k_3 \end{bmatrix} \text{ and } \bm{V}^k = \begin{bmatrix} V^k_1 \\ V^k_2 \\ V^k_3 \end{bmatrix}$$
% The mean positions and speeds vector and the covariance matrix at $t_k$ in $RSW$ frame are respectively noted $\bm{\mu}_k$ and $\Sigma^k$ as follows:
% $$ \bm{\mu}^k = \begin{bmatrix} \mu^k_1 \\ \mu^k_2 \\ \mu^k_3 \\ \mu^k_4 \\ \mu^k_5 \\ \mu^k_6 \end{bmatrix} \text{ and } \Sigma = \begin{bmatrix}\Sigma^k_R & 0_{3,3} \\ 0_{3,3} & \Sigma^k_V \end{bmatrix} \text{ with } \Sigma^k_V \text{ and } \Sigma^k_R \text{ the positions and speeds covariance matrices} $$
%
% \section*{Useful Stuff}
% $$
%  \phi_{HCW}(t) =
%  \begin{bmatrix}
%   4-3\cos(nt)    & 0 & 0          & \frac{\sin(nt)}{n}      & \frac{2(1-\cos(nt))}{n} & 0                  \\
%   6(\sin(nt)-nt) & 1 & 0          & \frac{2(\cos(nt)-1)}{n} & \frac{4\sin(nt)-3nt}{n} & 0                  \\
%   0              & 0 & \cos(nt)   & 0                       & 0                       & \frac{\sin(nt)}{n} \\
%   3n\sin(nt)     & 0 & 0          & \cos(nt)                & 2\sin(nt)               & 0                  \\
%   6n(\cos(nt)-1) & 0 & 0          & -2\sin(nt)              & 4\cos(nt)-3             & 0                  \\
%   0              & 0 & -n\sin(nt) & 0                       & 0                       & \cos(nt)           \\
%  \end{bmatrix}
% $$
%
%
%
% $$
%  \text{erf(x)} = \frac{2}{\sqrt{\pi}} \int_0^x e^{-s^2} \mathrm ds
% $$
%
%
% $$
%  \mathbb{P}(Y_k \in [-R,R]^3 ) = \frac{1}{(2\pi)^\frac{3}{2} \prod_{l=1}^3 \sigma_l} \int_{-R}^{R} \int_{-R}^{R} \int_{-R}^{R} \exp \left (-\sum_{l=1}^3 \frac{(y_l - \mu_{Y_k}^{(l)})^2}{2 \sigma_l^2} \right ) \mathrm dy_1 \mathrm dy_2 \mathrm dy_3
% $$
%
% \section{Spherical model and bounding box}
% \subsection{Spherical model}
% We use a spherical model so that taking into account the rotation of the objects is no longer necessary. First we define two spheres, of radii $R_p$ and $R_s$ for the primary and secondary objects respectively, taken as the maximum span length of each object as seen from their own inertial center. Thus, the collision event occurs whenever one sphere encounters the other, that is if there exists $k$ such that the relative position of the objects $Y^k$ is lesser than $R_p + R_s$. We can then define the collision domain, a sphere of radius $R=R_p + R_s$.
% \subsection{Bounding box}
% Integrating on a sphere is quite difficult, so for a first approach we define a bounding box that contains the collision domain (the sphere). To calculate the probability of collision between both objects,
% we make a change of variables into the frame generated by the eigenvectors of the relative position covariance matrix. We denote the associated eigenvalues by $\sigma_l^2$, $l \in \Iintv{1}{3}$. Setting the uniform bounding box of length $2R$ in this new frame we obtain from \cite{serra:tel-01261497} that
% $$
%  \mathbb{P}(Y_k \in [-R,R]^3 ) = \frac{1}{(2\pi)^\frac{3}{2} \prod_{l=1}^3 \sigma_l} \int_{-R}^{R} \int_{-R}^{R} \int_{-R}^{R} \exp \left (-\sum_{l=1}^3 \frac{(y_l - \hat{\mu}_{Y_k}^{(l)})^2}{2 \sigma_l^2} \right ) \mathrm dy_1 \mathrm dy_2 \mathrm dy_3
% $$
%
% \section{Meeting Modeling}
% \subsection{relative dynamics}
% The first step we need to do is to change the reference frame. In fact, the datas and the measurements given by Cspoc have been calculated in the terrestrial frame. However, it is more efficient to analyse this datas in a new frame centered on the primary object. \\ \\
%
% From a vector $\bm{X} = \begin{bmatrix} \bm{r} \\ \bm{v} \end{bmatrix}_{ECI} \in M_{6,1}(\mathbb{R}) \text{where} \ \bm{r} \in M_{3,1}(\mathbb{R}) \text{and} \ \bm{r} \in M_{6,1}(\mathbb{R})$ taken in ECI frame, we define the three vectors $\bm{R}$,$\bm{S}$ and $\bm{W}$ as follows. \\ \\
%
% $$
% \boxed{ \bm{R} = \frac{\bm{r}}{\norm{\bm{r}}} } \ \ \
% \boxed{ \bm{W} = \frac{\bm{r} \wedge \bm{v}}{\norm{\bm{\bm{r} \wedge \bm{v}}}} } \ \ \
% \boxed{ \bm{S} = \bm{W} \wedge \bm{R} } \ \ \
% $$ \\ \\
%
% We also define the transition matrix $P_{ECI -> RSW}$ from $RSW$ to $ECI$ by:
%
% $$
%  P_{ECI -> RSW} =
%  \begin{bmatrix}
%   \bm{R} & \bm{S} & \bm{W}
%  \end{bmatrix}^t
% $$ \\ \\
%
% Now, we need Bour's formula to convert the objects speeds from $ECI$ to $RSW$ frame. It gives:
% $$
%  \left[ \frac{ \mathrm d \mathrm \bm{\vec {B}}}{ \mathrm dt} \right ]_{(ECI)}=\left [ \frac{ \mathrm d \mathrm \bm{ \vec{B}}}{ \mathrm dt}  \right ]_{(RSW)}+ \vec{ \Omega}_{(RSW/ECI)} \wedge \mathrm \bm{\vec {B}} $$
%
% $$ \text{where} \
% \begin{cases}
% 	\vec \Omega \ \text{is the angular speed vector} \\
% 	\bm \vec {B} \ \text{is some vector}
% \end{cases}
% $$ \\ \\
%
% $\bm \vec {\Omega}$ is defined as follows:
% $$
% \bm \vec {\Omega} = \frac{\bm{r_p} \wedge \bm{v_p}}{\norm{\bm{r_p}}^2}
% $$
%
% We put $O$,$P$ and $S$ the frame's center, the primary object and the secondary object respectively.
% We get:
%
% $$
%  \left[ \frac{ \mathrm d \mathrm \bm{\vec {OP}}}{ \mathrm dt} \right ]_{(ECI)} = \bm{r_p^{tca}} \quad \quad \quad
%  \left[ \frac{ \mathrm d \mathrm \bm{\vec {OS}}}{ \mathrm dt} \right ]_{(ECI)} = \bm{r_s^{tca}}
% $$\\
% And we want to find
% $$
%  \left[ \frac{ \mathrm d \mathrm \bm{\vec {PS}}}{ \mathrm dt} \right ]_{(RSW)}
% $$\\
% We have:
% $$
%  \left[ \frac{ \mathrm d \mathrm \bm{\vec {OS}}}{ \mathrm dt} \right ]_{(ECI)} = \left[ \frac{ \mathrm d \mathrm \bm{\vec {OP}}}{ \mathrm dt} \right ]_{(ECI)} + \left[ \frac{ \mathrm d \mathrm \bm{\vec {PS}}}{ \mathrm dt} \right ]_{(ECI)}
% $$\\
% By using Bour's Formula, we find:
% $$
%  \left[ \frac{ \mathrm d \mathrm \bm{\vec {OS}}}{ \mathrm dt} \right ]_{(ECI)} = \left[ \frac{ \mathrm d \mathrm \bm{\vec {OP}}}{ \mathrm dt} \right ]_{(ECI)} + \left [ \frac{ \mathrm d \mathrm \bm{ \vec{PS}}}{ \mathrm dt}  \right ]_{(RSW)}+ \vec{ \Omega}_{(RSW/ECI)} \wedge \mathrm \bm{\vec {PS}}
% $$\\
% Thus:
% $$
% \boxed{
%   \left [ \frac{ \mathrm d \mathrm \bm{ \vec{PS}}}{ \mathrm dt}  \right ]_{(RSW)} = \left[ \frac{ \mathrm d \mathrm \bm{\vec {OS}}}{ \mathrm dt} \right ]_{(ECI)} - \left[ \frac{ \mathrm d \mathrm \bm{\vec {OP}}}{ \mathrm dt} \right ]_{(ECI)} - \vec{ \Omega}_{(RSW/ECI)} \wedge \mathrm \bm{\vec {PS}}
% \quad \quad \quad (1)}
% $$ \\ \\
% We recall that:
% $$
% \bm{Y} =
% \begin{bmatrix}
% 	\bm{r_s - r_p} \\
% 	\bm{v_s - v_p}
% \end{bmatrix}_{ECI}
% \ \ \ \
% \bm{X_p} =
% \begin{bmatrix}
% 	\bm{r_p} \\
% 	\bm{v_p}
% \end{bmatrix}_{ECI}
% \ \ \ \
% \bm{X_s} =
% \begin{bmatrix}
% 	\bm{r_s} \\
% 	\bm{v_s}
% \end{bmatrix}_{ECI}
% \ \ \ \
% P_{ECI -> RSW} =
% 	\begin{bmatrix}
%   	\bm{R} & \bm{S} & \bm{W}
% \end{bmatrix}^t
% $$\\
% These notations involve:
% $$(1) \Leftrightarrow
% \boxed{
% 	\begin{pmatrix}
% 		\bm{P_{ECI -> RSW}} & \bm{0_{3,3}} \\
% 		-\bm{P_{ECI -> RSW}} \bm{\Omega} & \bm{P_{ECI -> RSW}}
% 	\end{pmatrix}
% 	\begin{bmatrix}
% 		\bm{r_s - r_p} \\
% 		\bm{v_s - v_p}
% 	\end{bmatrix}
% 	=
% 	\begin{bmatrix}
% 		\bm{P_{ECI -> RSW}} (\bm{r_s - r_p}) \\
% 		\bm{P_{ECI -> RSW}} (\bm{v_s - v_p}) -\bm{\Omega} (\bm{r_s - r_p})
% 	\end{bmatrix}
% }
% $$



\newpage
\bibliographystyle{apalike}
\bibliography{references}
\end{document}
